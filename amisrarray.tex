
\documentclass[draft,ras]{Template/AGUTeX}
\usepackage{graphicx}   % if you want to include graphics files
\usepackage{subfigure}
\usepackage{setspace}
\usepackage{amsxtra}
\usepackage{amsmath}
\usepackage{amssymb}
\usepackage{multirow}
\usepackage{bm}
\RequirePackage{lineno}
\linenumbers
% graphics path
\graphicspath{{Figs/}}

\authorrunninghead{WICHMAN ET AL.}
\titlerunninghead{AMISR ARRAY}


\authoraddr{Corresponding author: A. R. Wichman,
Department of Electrical \& Computer Engineering,
Boston University, 8 Saint Mary�s Street
Boston, MA 02215, USA.
(arw@bu.edu)}

\begin{document}

\title{AMISR Array Antenna Patterns And Mutual Coupling Estimates}%
%%%%%%%%%%%%%% Author Info %%%%%%%%%%%%%%%%%%%%%%%%%%%%%%%%%%%%%
\authors{Adam Whichman, \altaffilmark{1} John Swoboda,\altaffilmark{1}
Joshua Semeter,\altaffilmark{1} }

\altaffiltext{1}{Department of Electrical \& Computer Engineering,
Boston University, Boston, Massachusetts, USA.}
\altaffiltext{2}{Atmospheric Science Division,
MIT Haystack Observatory, Westford Massachusetts, USA.}
\begin{abstract}
This reports results from modeling the far field radiation patterns for various proposed configurations of the Advanced Modular Incoherent Scatter Radar (AMISR) phased array, ranging up to 4,096 cross half-wavelength dipoles in a 8x16 unit panel configuration presently installed at Poker Flat, Alaska.  The boresight far field radiation was determined using the multi level fast monopole method (MLFMM), a variation on the method of moments, with the FEKO suite from EM Software \& Systems.  The normalized antenna radiation intensity and antenna patterns are compared to the analytical result from ideal element and array patterns, at two azimuthal orientations, taking the difference as a proxy for mutual coupling.  In each case the idealized element and array pattern (without backplane) overestimates the far field radiation and normalized radiation intensity when the elevation drops 20 or more degrees from boresight.  While the difference between the numerical solution and the ideal normalized radiation intensity can span up to 20 dB, however, given the high centerline gain this variation arises at points where the gain or radiation pattern amplitude has already fallen significantly (e.g., more than 40 dB for the 1x32 array, more than 35 dB for the 8x16 array) from center-peak.  Mutual coupling thus appears to have minimal impact on boresight operation, but may warrant further investigation for scanning angles approaching ~20+ degrees, depending on the array configuration under study.  
\end{abstract}


\begin{article}

\section{Introduction}
The Advanced Modular Incoherent Scatter Radar (AMISR) is a modular phased array for atmospheric research at 430-450 MHz. [1]  The current array configurations include a 8x16 panel face in Poker Flat, Alaska and two faces in Resolute Bay, Canada, with ~ 1.1$^{\circ}$ beamwidth, an aperture ~ 715 m$^2$, and gain ~ 43 dBi. [1]

Several other antenna configurations have been proposed.  The antenna radiation and radiation intensity patterns, however, have not been definitively determined.  As a result, it is not known how the predictive antenna pattern should be altered from the idealized case to account for mutual coupling.  In this project we investigate the far field antenna and normalized radiation intensity patterns (�antenna power�) using numerical simulation to solve Maxwell�s equations for a realistic array model.  Because the numerical solution includes mutual coupling, we can estimate the mutual coupling effects by comparing the numerical solution to the respective ideal antenna patterns.  

For large arrays mutual coupling is generally addressed by assuming it affects scale but not pattern.  In large arrays with elements spaced along a grid at regular intervals, the edge effects are ignored and the pattern distortion effect is a simple scaling of relative amplitude patterns while preserving the relative pattern shape. [2] In the AMISR case, we find that edge effects are insignificant within 20$^{\circ}$of boresight but become increasingly noticeable at lower elevation for each configuration studied.  The edge effects, however, correspond to reduced far field gain at low elevation where the gain has already fallen by a sizeable margin from the maximum.  This result may matter to operations that steer the beam at or above ? = 20� from boresight elevation.   

\section{Methodology}
\subsection{Mutual Coupling}
The dense AMISR antennas are subject to mutual coupling among the constituent dipole elements from radiation, electronic cross talk, and indirect scattering from proximate objects.  This report considers the effects of radiative mutual coupling, in which each antenna radiates some portion of received signal to each other antenna element, and likewise receives radiation from each other antenna element as well as from the original source. [2, 3, 4, 5] The variation in the antenna currents results in variations in the element-wise input impedances as well as gain and the far field radiation patterns. The mutual coupling can also excite the polarization not directly excited by the signal source.  [2, 3]  In this project, we focus on the far field radiation pattern amplitude, without considering the detailed affects on element impedance or associated scattering matrix.  Polarization data is available and could be the basis of further study.  

\subsection{Antenna Pattern}
Active impedance of an array element is difficult to measure and its availability for the proposed AMISR configurations unclear. [3, 4, 5, 6]  In response to this general problem, several predictive numerical techniques have been developed to estimate far field patterns including mutual coupling. [3, 5, 6] By modeling the case of uniform element excitation this study conceptually follows the �average active-element model� insight that groups mutual coupling effects with an �average� antenna element pattern rather than in the element excitation current.  [3, 4]  We do not attempt to calculate an average element pattern, however, but simply compare the numerical model, with mutual coupling, to the ideal antenna pattern.  In particular, we take the idealized cross half wave dipole antenna element radiation intensity as $E_a=1 + \cos^2(\theta)$, and model the array pattern as two interleaved rectangular arrays with 2 dx spacing and a $dx$, $dy/2$ relative offset.  [3, 8]  The ideal comparison does not account for reflection from the backplane.

In numerical antenna simulation the method of moments (MoM) takes mutual coupling into account when solving the applicable Maxwell equations. [3, 5, 6]  While the MoM has been widely used in optimization and design analysis codes like NEC, the method is cumbersome for electrically large problems.  [7, 10] A variation on the MoM formulation, the multipole fast monopole method (MLFMM), is better suited to far field calculations for large structures such as the various proposed AMISR configurations. [9, 10] Generally speaking, the MLFMM formulation uses characteristic basis functions and takes advantage of symmetry / sparsity in the matrix formulation for calculating the full-wave current-based solution of Maxwell�s equations.  [Id.] The MLFMM solves the interaction between groups of basis functions rather than individual basis functions.  [Id.] This provides more tractable $N \log(N)$ memory scaling, rather than $N\log^2(N)$ scaling with the method of moments formulation. [Id.]  

Commercial antenna design software provides a MLFMM implementation.  For this project we use the FEKO suite 6.3 from EM Software \& Systems as the computer-aided design (CAD) package for antenna analysis.  The numerical routines include a MLFMM solver. [9]

\subsection{CAD Testbed}
The CAD model is assembled in FEKO using the array and element parameters provided by the AMISR operator SRI International. [1]
\subsubsection{AMISR unit panel}
The AMISR unit panel comprises 32 cross dipoles in triangular configuration on roughly a 3.47 m x 1.98 m meshed aluminum backplane. [1]  The AMISR operating range is 430 � 450 MHz, or free space wavelengths 67 � 70 cm.  [1] The AMISR unit panel elements comprise crossed 34 cm long, 1 cm radius aluminum dipoles, or slightly less than half-wave at 450 MHz.  [11] The crossed dipoles are situated 18 cm above the backplane, each with its own solid-state power supply. [11]

The long (x) axis element (column) spacing dx = 0.4343 m, while the short (y) axis element (row) spacing dy=0.4958 m.  Every second column is (row) offset dy/2 relative to the prior column, so that the element row for each second column is located midway between the prior column rows.  This allows us to model the unit panel as two interleaved, finite 4x4 arrays with the second array offset by (dx, dy/2) from the first.  Notwithstanding the spacing adjustments available when combining unit panels into array configurations, as shown below, given the relative dipole length (34 cm) and y spacing (dy = 49.58 cm) this results in roughly 6 cm y-directed dipole overlap between panels. 

The dipole overlap in the AMISR arrays requires a finite array CAD model (e.g., precludes modeling subsets of infinite arrays with periodic boundary conditions), and limits us to replicating core elements without associated finite ground planes.  Nonetheless, the AMISR unit panel symmetry lets us generate a CAD model from a base configuration using two offset cross dipoles above an infinite, perfectly conducting ground plane. 

The dipoles are specified in aluminum using default material parameter files (e.g., $\mu_r$ = 1, $\sigma$ = 3.816e7 S/m) in FEKO, while the vertical post is treated as a perfect conductor.  The dipoles and vertical post are electrically isolated.  We separately excite the x and y oriented dipoles with ports on each element.  The y-oriented dipoles are excited at a 90$^{\circ}$ phase offset from the x-oriented dipoles.  Because we simulate boresight far fields, positional or steering phase offsets are not introduced in the excitation.  The far field simulation uniformly excites each port by 1 V at 440 MHz with 50 $\Omega$ source impedance.  The dipole impedance, of course, varies with mutual coupling in each element as numerically determined in simulation.

The AMISR unit panel is then modeled as a 4x4 finite array of the two-element CAD model, with x offset 2 dx and y offset dy.  

The AMISR unit panel CAD model is scaled by m and n for m x-oriented panels, and n y-oriented panels to create the CAD model for a m x n panel array.  The ground plane is imposed after scaling the two-element CAD model for the desired array configuration.

The AMISR unit panel CAD model is scaled by m and n for m x-oriented panels, and n y-oriented panels to create the CAD model for a m x n panel array.  The ground plane is imposed after scaling the two-element CAD model for the desired array configuration.

\subsubsection{Hardware}

The FEKO suite 6.3 was installed on a 2.6 GHz Intel 64 bit processor with 256 GB RAM.  Subject to the FEKO trial license, we used only 12 of 48 cores for the simulations.  This proved sufficient for simulations up to 4,096 cross dipole elements.  The primary limitation in data manipulation proved to be the graphical interface when formulating the test structure or formatting the far field data solution display, and this might be addressed in future simulations by scripting the CADFEKO or POSTFEKO operations or using a cluster with larger RAM.    

\subsection{Simulation}

We investigate the antenna patterns for three array configurations: 2x4, 1x32, and 8x16 panels.  A 12x24 simulation was not completed due to hardware limitations. 

The far field total gain and radiation pattern were solved in 3D in the half space above the ground plane for 440 MHz without taper.  Given the tight 3dB bandwidth, sub-degree increments (0.1 $^{\circ}$) were chosen for the elevation angle $\{\theta\}$ in the 1x32 and 8x16 (but not 2x4) array configurations, while the azimuth $\{\phi\}$ was solved at 1-5$^\circ$ increments.  The results at $\phi = 0, 90^\circ$ are investigated, although the resulting far field simulation data spans the entire half-space above the ground plane. 

Ideal antenna patterns for each given configuration were also computed, without ground plane, using Matlab (see Appendix, Section IV). [3] The difference between the numerical simulation in FEKO, and the ideal antenna pattern from Matlab, is taken as a proxy for mutual coupling in the antenna. 

\section{Results}

The simulation results are provided in the appendix (Sections I-III), and accompanying data files.  The 8x16 simulations recover the published AMISR gain and beamwidth, suggesting reasonable agreement on boresight between the CAD model and the AMISR design specification.  For exposition the 1x32 array results are considered more closely.  As shown in Figs. 7 and 9, the ideal antenna power pattern (normalized radiation intensity) settles into a series of lobes with roughly constant max value greater than � 20 dBi ($\phi = 0^\circ$) and -50 dBi ($\phi = 90^\circ$) for $|\theta| > 30^\circ$.

The numerical solution (with mutual coupling) continues to fall, however, as elevation comes down off boresight, differing in some instances by more than 20 dBi from the idealized pattern.  For $|\theta|\ge 20^\circ$ this amounts to a $\theta$ dependent reduction in relative amplitude for the antenna power pattern, while preserving the relative pattern shape (for $\theta < 80^\circ$).  

Figures 8 and 10 show the corresponding normalized radiation patterns (ideal and by MLFMM) for $\phi=0, 90^\circ$. As suggested by both the gain and pattern plots, the edge effects diminish with center lobe beamwidth.  Given the high center-lobe gain, the coupling effects may be insignificant on boresight, or even desirable insofar as they reduce off-center detection.  In the $\phi = 0^\circ$ orientation, for example, the calculated edge effect (or, more precisely, the actual radiation intensity pattern) suppresses the third and fourth sidelobe levels at $|\theta|$ = 40 or 60$^\circ$.  A $\theta$-dependent average active-element pattern might be extracted from the ratio of normalized antenna radiation patterns for a range of $\phi$, see Appendix, Section II.B, but in general the difference attributable to edge effects at low elevation for the desired low beamwidth $\phi = 90^\circ$ scan is marginal. 

Similar results can be seen in the Appendix for the 2x4 and 8x16 configurations.

\section{Conclusions}

Mutual coupling appears to have marginal impact on far field patterns for boresight operation in the three configurations investigated at $\phi=0$ or 90$^\circ$.  The primary effect is a conceptually beneficial reduction in gain and antenna pattern off beam center, at low elevations.  The edge effects appear at elevation greater than 20$^\circ$ from boresight.  This suggests that mutual effects may be more significant for beam steering at more than 20$^\circ$ off center, and may warrant further study by numerical techniques set forth in this report.

\section{Future Simulations}
The far field simulations provide richer results than explored in this summary report.  For example, the results include data on the far field polarization that has not been explored.  That information could be investigated further, in comparison with the idealized array as needed to isolate mutual coupling effects.

Future simulations for AMISR array configurations could examine mutual coupling affects on antenna patterns for beam steering near the grating lobe limits.  Given an antenna array configuration, the relevant phase delays could be manually added between dipole elements.  Because each cross dipole consists of two ports, this manual approach would entail assigning phase delays for twice as many sources as cross dipole antenna elements.  It may be straightforward to script this phase delay specification.  Alternatively, the simulation could specify source phase delays / amplitude variations by loading suitable data files from actual AMISR runs for the steering angles of interest.  This will require some greater facility with the FEKO suite, as well as more data files from SRI International, but should be tractable.  Depending on the array configurations and number of elements, as well as the hardware used, each steering angle may entail several hours for the associated far field radiation pattern calculations.  FEKO scripting capability, however, probably allows for largely automating the problem setup. 

\appendix
\section{Derivation of Idealized AMISR Array Pattern} \label{App:AMISRarr}
The current antenna on the AMISR systems is made up 8x16 set of panel of half wave cross dipoles. Each panel has 32 cross dipoles in a 8x4 hexagonal configuration. In the current set up at the Poker Flat site this yields at 4096 element array in a 64x64 element hexagonal configuration.

In order to simplify the antenna can be treated as two rectangular arrays of cross dipoles interleaved together. In the $x$ direction each of these arrays will have a spacing of $2d_x$ with $M/2$ elements. The $y$ direction will be of length $N$ elements and spacing $d_y$. Using basic planar phase array theory, \citep{Balanis:2005:ATA:1208379}, we can start with the linear array pattern from the first array can be represented as 

\begin{equation}
\label{eqn:arrpat1}
E_1(\theta,\phi) =\displaystyle \sum_{m=1}^{M/2}\sum_{n=1}^{N} e^{-j2\left(m-1\right)kd_x\sin\theta\cos\phi -j\left(n-1\right) k d_y\sin\theta\sin\phi}.
\end{equation}

\noindent Since the second array can be though of a shifted version of the first in the $x$ direction we get the following

\begin{equation}
\label{eqn:arrpat2}
E_2(\theta,\phi) =\displaystyle \sum_{m=1}^{M/2}\sum_{n=1}^{N} e^{-j\left(2m-1\right)kd_x\sin\theta\cos\phi -j\left(n-1/2\right) k d_y\sin\theta\sin\phi}.
\end{equation}

In order to simplify notation we will make the following substitutions, $\psi_x = -k d_x\sin\theta\cos\phi$, $\psi_y = -k d_y\sin\theta\sin\phi$. Using Equations \ref{eqn:arrpat1} and \ref{eqn:arrpat2} we can see the following relationship,

\begin{equation}
\label{eqn:arrpateqn}
E_2(\theta,\phi)  = e^{j(\psi_y/2 + \psi_x)} E_1(\theta,\phi)  = \displaystyle \sum_{m=1}^{M/2}\sum_{n=1}^{N} e^{-j2\left(m-1\right) \psi_x -j\left(n-1\right) \psi_y}.
\end{equation}

\noindent Adding $E_1$ and $E_2$ together we get the following linear array pattern

\begin{equation} \label{eq1}
\begin{split}
E(\theta,\phi) &= \displaystyle  \left(1+e^{j(\psi_y/2 + \psi_x)}\right)\sum_{m=1}^{M/2}\sum_{n=1}^{N} e^{-j2\left(m-1\right) \psi_x -j\left(n-1\right) \psi_y}.\\
& = \frac{1}{MN} \left(1+e^{j(\psi_y/2 + \psi_x)}\right)\frac{\sin((M/2) \psi_x)}{\sin(\psi_x)} \frac{\sin((N/2) \psi_x)}{\sin(\psi_x/2)}.
\end{split}
\end{equation}

Since the array is steerable this can be taken into account in the equations by simply changing the definitions of $\psi_x $ and $\psi_y$ to $\psi_x = k d_x(\sin\theta\cos\phi-\sin\theta_s\cos\phi_s)$, and $\psi_y = k d_y(\sin\theta\sin\phi-\sin\theta_s\sin\phi_s)$. Lastly the antenna pattern of a single cross dipole can be represented as $ \frac{1}{2}(1+\cos(\theta)^2)$\citep{Balanis:2005:ATA:1208379}. By taking the squared magnitude of the array factor and multiplying it with the pattern of the dipole we get Equation \ref{eqn:amisrpat},

\begin{equation}
\label{eqn:amisrpatfinal}
F(\theta_s,\phi_s,\theta,\phi) = \frac{1}{2}(1+\cos(\theta)^2)\left| \frac{1}{MN} \left(1+e^{j(\psi_y/2 + \psi_x)}\right)\frac{\sin((M/2) \psi_x)}{\sin(\psi_x)} \frac{\sin((N/2) \psi_x)}{\sin(\psi_x/2)}\right|^2.
\end{equation}
 


\begin{acknowledgments}
(Text here)
\end{acknowledgments}

\bibliographystyle{BibTeX/agufull08}
\bibliography{BibTeX/Amisrarray}
\end{article}

\end{document}

%%%%%%%%%%%%%%%%%%%%%%%%%%%%%%%%%%%%%%%%%%%%%%%%%%%%%%%%%%%%%%%

More Information and Advice:

%% ------------------------------------------------------------------------ %%
%
%  SECTION HEADS
%
%% ------------------------------------------------------------------------ %%

% Capitalize the first letter of each word (except for
% prepositions, conjunctions, and articles that are
% three or fewer letters).

% AGU follows standard outline style; therefore, there cannot be a section 1 without
% a section 2, or a section 2.3.1 without a section 2.3.2.
% Please make sure your section numbers are balanced.
% ---------------
% Level 1 head
%
% Use the \section{} command to identify level 1 heads;
% type the appropriate head wording between the curly
% brackets, as shown below.
%
%An example:
%\section{Level 1 Head: Introduction}
%
% ---------------
% Level 2 head
%
% Use the \subsection{} command to identify level 2 heads.
%An example:
%\subsection{Level 2 Head}
%
% ---------------
% Level 3 head
%
% Use the \subsubsection{} command to identify level 3 heads
%An example:
%\subsubsection{Level 3 Head}
%
%---------------
% Level 4 head
%
% Use the \subsubsubsection{} command to identify level 3 heads
% An example:
%\subsubsubsection{Level 4 Head} An example.
%
%% ------------------------------------------------------------------------ %%
%
%  IN-TEXT LISTS
%
%% ------------------------------------------------------------------------ %%
%
% Do not use bulleted lists; enumerated lists are okay.
% \begin{enumerate}
% \item
% \item
% \item
% \end{enumerate}
%
%% ------------------------------------------------------------------------ %%
%
%  EQUATIONS
%
%% ------------------------------------------------------------------------ %%

% Single-line equations are centered.
% Equation arrays will appear left-aligned.

Math coded inside display math mode \[ ...\]
 will not be numbered, e.g.,:
 \[ x^2=y^2 + z^2\]

 Math coded inside \begin{equation} and \end{equation} will
 be automatically numbered, e.g.,:
 \begin{equation}
 x^2=y^2 + z^2
 \end{equation}

% IF YOU HAVE MULTI-LINE EQUATIONS, PLEASE
% BREAK THE EQUATIONS INTO TWO OR MORE LINES
% OF SINGLE COLUMN WIDTH (20 pc, 8.3 cm)
% using double backslashes (\\).

% To create multiline equations, use the
% \begin{eqnarray} and \end{eqnarray} environment
% as demonstrated below.
\begin{eqnarray}
  x_{1} & = & (x - x_{0}) \cos \Theta \nonumber \\
        && + (y - y_{0}) \sin \Theta  \nonumber \\
  y_{1} & = & -(x - x_{0}) \sin \Theta \nonumber \\
        && + (y - y_{0}) \cos \Theta.
\end{eqnarray}

%If you don't want an equation number, use the star form:
%\begin{eqnarray*}...\end{eqnarray*}

% Break each line at a sign of operation
% (+, -, etc.) if possible, with the sign of operation
% on the new line.

% Indent second and subsequent lines to align with
% the first character following the equal sign on the
% first line.

% Use an \hspace{} command to insert horizontal space
% into your equation if necessary. Place an appropriate
% unit of measure between the curly braces, e.g.
% \hspace{1in}; you may have to experiment to achieve
% the correct amount of space.


%% ------------------------------------------------------------------------ %%
%
%  EQUATION NUMBERING: COUNTER
%
%% ------------------------------------------------------------------------ %%

% You may change equation numbering by resetting
% the equation counter or by explicitly numbering
% an equation.

% To explicitly number an equation, type \eqnum{}
% (with the desired number between the brackets)
% after the \begin{equation} or \begin{eqnarray}
% command.  The \eqnum{} command will affect only
% the equation it appears with; LaTeX will number
% any equations appearing later in the manuscript
% according to the equation counter.
%

% If you have a multiline equation that needs only
% one equation number, use a \nonumber command in
% front of the double backslashes (\\) as shown in
% the multiline equation above.

%% ------------------------------------------------------------------------ %%
%
%  SIDEWAYS FIGURE AND TABLE EXAMPLES
%
%% ------------------------------------------------------------------------ %%
%
% For tables and figures, add \usepackage{rotating} to the paper and add the rotating.sty file to the folder.
% AGU prefers the use of {sidewaystable} over {landscapetable} as it causes fewer problems.
%
% \begin{sidewaysfigure}
% \includegraphics[width=20pc]{samplefigure.eps}
% \caption{caption here}
% \label{label_here}
% \end{sidewaysfigure}
%
%
%
% \begin{sidewaystable}
% \caption{}
% \begin{tabular}
% Table layout here.
% \end{tabular}
% \end{sidewaystable}
%
%

